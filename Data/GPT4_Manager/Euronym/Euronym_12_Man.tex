## Manager Version:

### Euronym Overview

**What is it?**
Euronym is a tool designed to display place names on digital maps, tailored for web applications focusing on Europe. Imagine you're zooming in and out of an online map of Europe; Euronym ensures that the city and place names you see are clear, legible, and adjusted to the zoom level. So, the closer you zoom in, the more detailed and denser the labels become.

**Why is it important?**
For any organization that uses or creates web-based maps of Europe, ensuring that place names are accurately and effectively displayed is crucial. Whether it's for tourism, logistics, urban planning, or marketing, Euronym offers a professional solution. It enhances user experience by ensuring that maps are readable and interactive.

**How does it work?**
Euronym employs a smart algorithm that automatically decides how big a label should be and whether it should be displayed based on how much a user has zoomed in or out. The tool takes into account the importance of each label, ensuring that essential places are prioritized.

**Where is the data from?**
The names and places displayed by Euronym come from reputable sources such as EuroGeographics and the European Commission's table of town names. This ensures accuracy and reliability.

**Can I customize it for my needs?**
Yes! If your organization has specific needs or wants to host its own version of Euronym, the tool is flexible enough to allow for that. You can select the specific features you need and host them on your server.

**Who's behind it?**
Euronym is a product of collaborative effort and has been made possible by contributors like Jgaffuri. It's been around since 2022 and is under the protection of the EuroGeographics Open Data Licence.

**Anything else I should know?**
If you ever need assistance or want to provide feedback, the Euronym team welcomes questions, suggestions, or even just a thumbs-up for encouragement!

**Please Note:**
While Euronym aims for accuracy, it does not represent the political stance or opinions of the European Union on boundaries or territorial claims.

---

*Hope this version provides a succinct and clear overview for non-technical managers!*