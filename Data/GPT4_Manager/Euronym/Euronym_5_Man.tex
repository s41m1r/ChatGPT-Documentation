**Manager Version**:

---

# Euronym Overview

### What is Euronym?
Euronym provides labels for place names to be used in web mapping applications, with a particular focus on European locations. These labels automatically adjust based on how much you zoom in or out, ensuring that maps remain clean and readable at all scales.

### Why is it Important?
When displaying maps on web applications, having too many labels or improperly sized labels can make the map cluttered or confusing. Euronym uses advanced algorithms to ensure that only the most relevant place names are shown at the right sizes, enhancing the user experience.

![Label Placement Example](/docs/overview.gif)

### Key Benefits:
1. **Scalability**: Adapts to various zoom levels ensuring the most relevant information is displayed.
2. **Optimized for Europe**: While it's designed with a European focus, it’s versatile enough to be used for other regions as well.
3. **Customizable**: You can select the geographical area and the detail level as per your need.
4. **Reliable Data Sources**: Uses authoritative datasets like the EuroRegionalMap for accurate place names.

### Where can I see an example?
Euronym is being utilized by the Gridviz library. You can view a [demonstration here](https://eurostat.github.io/gridviz/examples/labels_.html).

### Anything else I should know?
- The project has been active since 2022 and is currently in its second version.
- The labels are provided with due respect to copyright laws, and proper attribution is required when used.
- If you ever face issues or need help, there’s a support team ready to assist. Plus, you can contribute or give feedback to improve the project further.

### Bottom Line:
Euronym is an essential tool for anyone wanting to provide clear and interactive web maps, especially if the focus is on Europe. It simplifies the visual presentation of maps, ensuring they’re easy to understand, which can greatly enhance user satisfaction and engagement.

--- 

This adapted version provides a high-level overview of Euronym, focusing on its purpose and benefits without diving into the technicalities. The goal is to make it easy for non-technical managers to understand the significance and potential applications of the tool.