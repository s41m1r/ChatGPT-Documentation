## Manager Version:

---

### Euronym: Adaptive Place Name Labels for Multi-Scale Web Mapping in Europe

**Overview:**  
Euronym is a tool designed to provide effective labeling for maps, especially tailored for European regions. It cleverly adjusts the number, size, and density of place names displayed on maps based on the user's zoom level, ensuring the best visual experience and readability.

**Why is it important?**  
When you zoom in or out on online maps, it's essential for place names (like cities or landmarks) to be shown clearly without clutter. Too many labels can be overwhelming, while too few might not provide enough information. Euronym automates this process, ensuring that only the most relevant labels are displayed at any given zoom level.

**Key Features:**  

1. **Dynamic Scaling:** Labels adapt their size based on zoom levels. This ensures that place names are always clearly visible and are never too small or overwhelmingly large.
  
2. **Geographic Specificity:** Users can choose specific geographic entities, or cover the entire dataset of Europe. 
  
3. **Customizable Encoding and Resolution:** Depending on the visual requirements, users can select the desired encoding (like UTF or ASCII) and resolution.

4. **Compatibility with Other Tools:** Euronym seamlessly integrates with the [Gridviz](https://github.com/eurostat/gridviz/blob/master/docs/reference.md#showing-labels) library, a known web mapping tool.

**Business Benefits:**  

- **Improved User Experience:** A map that adjusts its labels for the best readability ensures a more pleasant and efficient user experience.
  
- **Flexibility:** Organizations can deploy their own version of Euronym, choosing only the features they need.
  
- **Comprehensive Coverage:** The tool leverages trusted datasets like the EuroRegionalMap and the European commission town names table, ensuring the accuracy and comprehensiveness of the labels.

**Support and Contribution:**  
Euronym welcomes feedback, and for those interested, they can contribute to its development or even just show their appreciation by giving it a star on its project page.

**Legal & Copyright:**  
Usage of Euronym comes with specific licensing terms, mainly from EuroGeographics. It's essential to give due credit, particularly the **© EuroGeographics 2022** attribution when using their data.

**Disclaimer:**  
Euronym remains neutral regarding country and territory statuses in Europe. Its labeling does not imply any official stance or opinion on behalf of the European Union on such matters.

--- 

This version is tailored for non-technical managers to understand the purpose, value, and high-level features of Euronym without delving into the technicalities.