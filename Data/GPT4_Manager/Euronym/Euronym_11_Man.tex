**Manager Version:**

---

### **Euronym: Dynamic Place Name Labels for Web Maps**

#### **Overview:**
Euronym is an advanced tool designed to dynamically showcase place names on web maps, primarily tailored for Europe. Its primary function is to adjust the size, selection, and density of the place name labels based on how much a user zooms in or out on a map. This ensures an uncluttered visual representation, especially in regions where places are closely located.

#### **Benefits & Purpose:**
1. **Enhanced User Experience:** Users can now have clear and readable place labels irrespective of the zoom level, ensuring the map looks clean and isn’t overwhelmed with text.
   
2. **Tailored for Europe:** While the application can be used for various geographies, its primary focus is Europe, allowing businesses and organizations in the region to benefit from optimized data.

3. **Integrated with Gridviz:** Euronym is a part of the Gridviz library, a trusted tool for visual representation. It amplifies Gridviz's capabilities to make map reading intuitive and user-friendly.

4. **Data Precision:** The tool uses reputable datasets like the EuroRegionalMap and the European commission town names table. This ensures the accuracy of place names and their positioning.

5. **Customizable Deployment:** Organizations can host Euronym on their servers, allowing them to have better control and ensure data security.

#### **Licensing and Copyright:**
Euronym uses data under the [EuroGeographics Open Data Licence](https://www.mapsforeurope.org/licence). All organizations leveraging Euronym must provide proper attribution to EuroGeographics.

#### **Note:**
Map representations are for visual aid and do not suggest or imply any political opinion or boundary delimitations.

---

By offering an enhanced user experience with clear visual data representation, Euronym adds value to any organization or business looking to provide interactive maps on their platform. It simplifies map readability and ensures a neat presentation irrespective of user interaction levels.