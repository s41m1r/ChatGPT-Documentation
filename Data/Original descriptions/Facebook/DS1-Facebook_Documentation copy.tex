Overview
The Graph API is the primary way to get data into and out of the Facebook platform. It's an HTTP-based API that apps can use to programmatically query data, post new stories, manage ads, upload photos, and perform a wide variety of other tasks.
The Graph API is named after the idea of a "social graph" — a representation of the information on Facebook. It's composed of nodes, edges, and fields. Typically you use nodes to get data about a specific object, use edges to get collections of objects on a single object, and use fields to get data about a single object or each object in a collection. Throughout our documentation, we may refer to both a node and edge as an "endpoint". For example, "send a GET request to the User endpoint".
HTTP
All data transfers conform to HTTP/1.1, and all endpoints require HTTPS. Because the Graph API is HTTP-based, it works with any language that has an HTTP library, such as cURL and urllib. This means you can use the Graph API directly in your browser. For example, requesting this URL in your browser...
https://graph.facebook.com/facebook/picture?redirect=false
... is equivalent to performing this cURL request:
curl -i -X GET "https://graph.facebook.com/facebook/picture?redirect=false"
We have also enabled the includeSubdomains HSTS directive on facebook.com, but this should not adversely affect your Graph API calls.

Host URL
Almost all requests are passed to the graph.facebook.com host URL. The single exception is video uploads, which use graph-video.facebook.com.

Access Tokens
Access tokens allow your app to access the Graph API. Almost all Graph API endpoints require an access token of some kind, so each time you access an endpoint, your request may require one. They typically perform two functions:
		They allow your app to access a User's information without requiring the User's password. For example, your app needs a User's email to perform a function. If the User agrees to allow your app to retrieve their email address from Facebook, the User will not need to enter their Facebook password for your app to get their email address.
		They allow us to identify your app, the User who is using your app, and the type of data the User has permitted your app to access.
Visit our access token documentation to learn more.

Nodes
A node is an individual object with a unique ID. For example, there are many User node objects, each with a unique ID representing a person on Facebook. Pages, Groups, Posts, Photos, and Comments are just some of the nodes of the Facebook Social Graph.
The following cURL example represents a call to the User node.
curl -i -X GET \
  "https://graph.facebook.com/USER-ID?access_token=ACCESS-TOKEN"
This request would return the following data by default, formatted using JSON:
{
  "name": "Your Name",
  "id": "YOUR-USER-ID"
}
Node Metadata
You can get a list of all fields, including the field name, description, and data type, of a node object, such as a User, Page, or Photo. Send a GET request to an object ID and include the metadata=1 parameter:
curl -i -X GET \
  "https://graph.facebook.com/USER-ID?
    metadata=1&access_token=ACCESS-TOKEN"
The resulting JSON response will include the metadata property that lists all the supported fields for the given node:
{
  "name": "Jane Smith",
  "metadata": {
    "fields": [
      {
        "name": "id",
        "description": "The app user's App-Scoped User ID. This ID is unique to the app and cannot be used by other apps.",
        "type": "numeric string"
      },
      {
        "name": "age_range",
        "description": "The age segment for this person expressed as a minimum and maximum age. For example, more than 18, less than 21.",
        "type": "agerange"
      },
      {
        "name": "birthday",
        "description": "The person's birthday.  This is a fixed format string, like `MM/DD/YYYY`.  However, people can control who can see the year they were born separately from the month and day so this string can be only the year (YYYY) or the month + day (MM/DD)",
        "type": "string"
      },
...

/me
The /me node is a special endpoint that translates to the object ID of the person or Page whose access token is currently being used to make the API calls. If you had a User access token, you could retrieve a User's name and ID by using:
curl -i -X GET \
  "https://graph.facebook.com/me?access_token=ACCESS-TOKEN"

Edges
An edge is a connection between two nodes. For example, a User node can have photos connected to it, and a Photo node can have comments connected to it. The following cURL example will return a list of photos a person has published to Facebook.
curl -i -X GET \
  "https://graph.facebook.com/USER-ID/photos?access_token=ACCESS-TOKEN"
Each ID returned represents a Photo node and when it was uploaded to Facebook.
    {
  "data": [
    {
      "created_time": "2017-06-06T18:04:10+0000",
      "id": "1353272134728652"
    },
    {
      "created_time": "2017-06-06T18:01:13+0000",
      "id": "1353269908062208"
    }
  ],
}

Fields
Fields are node properties. When you query a node, or an edge, it returns a set of fields by default, as the examples above show. However, you can specify which fields you want returned by using the fields parameter and listing each field. This overrides the defaults and returns only the fields you specify, and the ID of the object, which is always returned.
The following cURL request includes the fields parameter and the User's name, email, and profile picture.
curl -i -X GET \
  "https://graph.facebook.com/USER-ID?fields=id,name,email,picture&access_token=ACCESS-TOKEN"
Data Returned
{
  "id": "USER-ID",
  "name": "EXAMPLE NAME",
  "email": "EXAMPLE@EMAIL.COM",
  "picture": {
    "data": {
      "height": 50,
      "is_silhouette": false,
      "url": "URL-FOR-USER-PROFILE-PICTURE",
      "width": 50
    }
  }
}
Complex Parameters
Most parameter types are straightforward primitives such as bool, string and int, but there are also list and object types that can be specified in the request.
The list type is specified in JSON syntax, for example: ["firstitem", "seconditem", "thirditem"]
The object type is also specified in JSON syntax, for example: {"firstkey": "firstvalue", "secondKey": 123}

Publishing, Updating, and Deleting
Visit our Facebook Sharing guide to learn how to publish to a User's Facebook or our Pages API documentation to publish to a Page's Facebook feed.
Some nodes allow you to update fields with POST operations. For example, you could update your email field like this:
curl -i -X POST \
  "https://graph.facebook.com/USER-ID?email=YOURNEW@EMAILADDRESS.COM&access_token=ACCESS-TOKEN"
Read-After-Write
For create and update endpoints, the Graph API can immediately read a successfully published or updated object and return any fields supported by the corresponding read endpoint.
By default, an ID of the object created or updated will be returned. To include more information in the response, include the fields parameter in your request and list the fields you want returned. For example, to publish the message “Hello” to a Page's feed, you could make the following request:
curl -i - X POST "https://graph.facebook.com/PAGE-ID/feed?message=Hello&
  fields=created_time,from,id,message&access_token=ACCESS-TOKEN"
The above code example is formatted for readability.
This would return the specified fields as a JSON-formatted response, like this:
{
  "created_time": "2017-04-06T22:04:21+0000",
  "from": {
    "name": "My Facebook Page",
    "id": "PAGE-ID"
  },
  "id": "POST_ID",
  "message": "Hello",
}
Refer to each endpoint's reference documentation to see if it supports read-after-write and what fields are available.
Errors
If the read fails for any reason (for example, requesting a non-existent field), the Graph API will respond with a standard error response. Visit our Handling Errors guide for more information.
You can usually delete a node, such as a Post or Photo node, by performing a DELETE operation on the object ID:
curl -i -X DELETE \
  "https://graph.facebook.com/PHOTO-ID?access_token=ACCESSS-TOKEN"
Usually you can only delete nodes that you created, but check each node's reference guide to see requirements for delete operations.

Webhooks
You can be notified of changes to nodes or interactions with nodes by subscribing to webhooks. See Webhooks.

Versions
The Graph API has multiple versions with quarterly releases. You can specify the version in your calls by adding "v" and the version number to the start of the request path. For example, here's a call to version 4.0:
curl -i -X GET \
  "https://graph.facebook.com/v4.0/USER-ID/photos
    ?access_token=ACCESS-TOKEN"
If you do not include a version number we will default to the oldest available version, so it's recommended to include the version number in your requests.
You can read more about versions in our Versioning guide and learn about all available versions in the Graph API Changelog.

Facebook APIs, SDKs, and Platforms
Connect interfaces and develop across platforms using Facebook's various APIs, SDKs, and platforms.

Next Steps
Get Started with Graph API – Let's explore the Facebook Social Graph using the Graph Explorer tool and run a couple requests to get data.
