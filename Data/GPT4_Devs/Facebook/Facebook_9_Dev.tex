**Developer Version**:

---

**Facebook Graph API Technical Documentation** 

---

**Introduction**:  
The Graph API is a core component for interfacing with Facebook's platform. It uses an HTTP-based structure, compliant with HTTP/1.1 standards and mandates HTTPS for all endpoints. Given its HTTP foundation, you can work with it using any language that offers an HTTP library (e.g., cURL, urllib).

**Endpoint Specifications**:
- **Host URL**: Mainly, utilize `graph.facebook.com`. For video uploads, switch to `graph-video.facebook.com`.
- **HTTP Library Compatibility**: Ideal for languages like cURL and urllib. 

---

**Access Tokens**:
- **Role**: They act as authentication keys for your app. Nearly every endpoint will demand one.
  1. Access a User's data without needing the User's password.
  2. Identify the app, User, and the scope of data the app can access.
- **Further Reading**: Check the detailed token documentation for deeper insights.

---

**Working with Nodes**:
- **Definition**: A node symbolizes an individual entity with a distinct ID, e.g., User, Page, Photo.
- **Querying a Node**:  
  ```bash
  curl -i -X GET "https://graph.facebook.com/USER-ID?access_token=ACCESS-TOKEN"
  ```
- **Metadata Fetching**: You can extract detailed information about a node's fields using:
  ```bash
  curl -i -X GET "https://graph.facebook.com/USER-ID?metadata=1&access_token=ACCESS-TOKEN"
  ```

---

**Special Endpoints**:
- **/me Node**: Represents the ID of the individual or Page using the access token.
  ```bash
  curl -i -X GET "https://graph.facebook.com/me?access_token=ACCESS-TOKEN"
  ```

---

**Edges**:
- **Definition**: Edges are connections or relationships between two nodes.
- **Example**: To retrieve a user's published photos:
  ```bash
  curl -i -X GET "https://graph.facebook.com/USER-ID/photos?access_token=ACCESS-TOKEN"
  ```

---

**Fields**:
- **Definition**: Fields are properties of nodes.
- **Custom Field Requests**: Override default return fields using the `fields` parameter:
  ```bash
  curl -i -X GET "https://graph.facebook.com/USER-ID?fields=id,name,email,picture&access_token=ACCESS-TOKEN"
  ```

---

**Handling Parameters**:
- **List Type**: Use JSON syntax: `["item1", "item2"]`.
- **Object Type**: Use JSON syntax: `{"key1": "value1", "key2": "value2"}`.

---

**CRUD Operations**:
- **Publishing**: Consult the Facebook Sharing guide.
- **Updating**: Example to modify the email field:
  ```bash
  curl -i -X POST "https://graph.facebook.com/USER-ID?email=NEW@EMAIL.COM&access_token=ACCESS-TOKEN"
  ```
- **Deleting**: Typically, you can remove nodes you've created. Example:
  ```bash
  curl -i -X DELETE "https://graph.facebook.com/NODE-ID?access_token=ACCESS-TOKEN"
  ```

---

**Read-After-Write**:
Some endpoints support immediate read operations post publishing or updating. By default, only the object's ID is returned, but you can request more:
```bash
curl -i -X POST "https://graph.facebook.com/PAGE-ID/feed?message=Hello&fields=created_time,from,id,message&access_token=ACCESS-TOKEN"
```

---

**Error Handling**:
A standard error response will be issued for failed reads. Check the Handling Errors guide for in-depth information.

---

**Webhooks**:
Stay updated on changes by subscribing to webhooks.

---

**API Versioning**:
- **Usage**: Append "v" followed by the version number at the start of the request path, e.g., `v4.0`.
- **Recommendation**: Always specify the version to avoid defaulting to the oldest version.
- **Additional Resources**: Consult the Versioning guide and Graph API Changelog.

---

**Related Tools**:
Harness the full power of Facebook's ecosystem with its range of APIs, SDKs, and platforms.

---

**What's Next?**:  
Dive deeper by utilizing the Graph Explorer tool. Test real-world scenarios to fetch and manipulate data.

---