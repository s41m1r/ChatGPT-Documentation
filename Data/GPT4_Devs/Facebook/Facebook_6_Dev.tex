**Developer Version:**

---

### **Facebook Graph API Overview for Developers**

#### **API Communication**
- **Protocol**: HTTP/1.1 
- **Security**: HTTPS required for all endpoints.
- **HTTP libraries**: Any language with an HTTP library (e.g., cURL, urllib) can interact with the Graph API.
  
#### **Host URL**
- **Standard Endpoint**: `graph.facebook.com`.
- **Video Endpoint**: `graph-video.facebook.com`.

#### **Access Tokens**
- **Purpose**:
  - Retrieve user data without needing the user's password.
  - Authenticate and authorize apps.
- All Graph API endpoints typically require an access token.
- Learn about token creation and management [here](#).

#### **Data Structure**
1. **Nodes**:
   - Represents objects like Users, Pages, Photos, etc.
   - Unique ID associated.
   - Example Call: `https://graph.facebook.com/USER-ID?access_token=ACCESS-TOKEN`
   - Metadata: `https://graph.facebook.com/USER-ID?metadata=1&access_token=ACCESS-TOKEN`

2. **Edges**:
   - Represents the connection between two nodes.
   - Example: Fetching photos connected to a user.
   - Example Call: `https://graph.facebook.com/USER-ID/photos?access_token=ACCESS-TOKEN`

3. **Fields**:
   - Represents properties of nodes.
   - Default fields can be overridden with the `fields` parameter.
   - Example Call: `https://graph.facebook.com/USER-ID?fields=id,name,email,picture&access_token=ACCESS-TOKEN`

4. **Special Nodes**:
   - `/me` endpoint: Represents the ID of the currently authenticated user or page.
   
#### **Complex Parameters**
- **List Type**: JSON syntax like `["firstitem", "seconditem", "thirditem"]`.
- **Object Type**: JSON syntax like `{"firstkey": "firstvalue", "secondKey": 123}`.

#### **Data Operations**
1. **Publishing**: Use the Facebook Sharing guide.
2. **Updating**: Use POST requests to nodes.
   - Example: Updating an email would be: `https://graph.facebook.com/USER-ID?email=YOURNEW@EMAILADDRESS.COM&access_token=ACCESS-TOKEN`
3. **Reading after Writing**: Immediate retrieval of data after create/update operations. 
   - Must specify the fields you want in the response.
   - Example Call: `https://graph.facebook.com/PAGE-ID/feed?message=Hello&fields=created_time,from,id,message&access_token=ACCESS-TOKEN`
4. **Deleting**: Use DELETE operation on node's object ID.
   - Typically, you can only delete nodes you created.
   - Example Call: `https://graph.facebook.com/PHOTO-ID?access_token=ACCESS-TOKEN`

#### **Webhooks**
- Get real-time notifications of changes or interactions with nodes.
- Refer to the Webhooks documentation [here](#).

#### **API Versions**
- Quarterly API version releases.
- Specify version by adding "v" followed by version number to the request path.
- Default version is the oldest available if not specified.
- Check the Versioning guide and Graph API Changelog.

#### **Integration**
- Utilize Facebook's range of APIs, SDKs, and platforms for more extensive development.

#### **Next Steps**
1. Familiarize yourself with the Graph Explorer tool.
2. Run test requests to get accustomed to data retrieval.

---

*Note: This developer version emphasizes technical specifications and call-to-action for developers to dive deep into the Graph API's functionalities.*