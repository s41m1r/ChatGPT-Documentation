### Developer Version:

---

# Euronym Developer Documentation

### Introduction

Euronym is a dynamic label placement tool tailored for multi-scale web mapping applications, concentrating predominantly on Europe. It harnesses an advanced [label placement algorithm](https://en.wikipedia.org/wiki/Automatic_label_placement) to adjust the density, selection, and size of labels conditional on the zoom level and label importance.

**Technical Insight**: A glance into the algorithm can be gleaned from the visualization [here](https://eurostat.github.io/gridviz/examples/labels_.html).

### API Details

- **Base URL**: `https://raw.githubusercontent.com/euronym/Nuts2json/master/pub/v2`

  **Alternative Deployment**: You can also self-host the data. Clone the repository and expose the `pub` folder via a web server. Modify the base URL accordingly, for example, to something like `https://www.mydomain.eu/path/to/pub/v2`.

- **URL Pattern**: `/<ENCODING>/<RESOLUTION>/<GEO>.csv`

  A practical instance would be: [`https://raw.githubusercontent.com/eurostat/euronym/main/pub/v2/UTF/100/LU.csv`](https://raw.githubusercontent.com/eurostat/euronym/main/pub/v2/UTF/100/LU.csv) which retrieves UTF-encoded data for a 100m/pixel resolution over Luxembourg.

  **Parameters**:

  - `ENCODING`: Defines data encoding (`UTF`, `ASCII`).
  - `RESOLUTION`: Indicates resolution detail, expressed in *m/pixel* (`20`, `50`, `100`, `200`).
  - `GEO`: Represents the geographic code. When set to `EUR`, a broader dataset is obtained.

  If further parameters or clarifications are needed, feel free to access the [issues page](https://github.com/eurostat/euronym/issues/new).

### Data Format

The returned data will be in a CSV structure, with columns indicating:

- `name`: Label text to be depicted on the map.
- `lon` & `lat`: Geographical coordinates in ETRS89 format. Consider these equivalent to WGS84.
- `rs` & `r1`: Control display resolution settings for labels.
- `cc`: Country code for filtering when `GEO` is set to `EUR`.

**Note**: For detailed representation settings based on `rs` and `r1`, refer to the main documentation above.

### Source Data & Transformation

- Euronym leans on datasets like [EuroRegionalMap](https://eurogeographics.org/maps-for-europe/euroregionalmap/) and the [European commission town names table](https://ec.europa.eu/regional_policy/en/information/maps/urban-centres-towns).
  
- The transformation logic can be found [here](https://github.com/eurostat/euronym/tree/main/src/), and it utilizes the [GeoTools library](https://www.geotools.org).

### Dependencies & Licensing

- Ensure the [EuroGeographics Open Data Licence](https://www.mapsforeurope.org/licence) terms are followed, especially when using the [EuroRegionalMap](https://eurogeographics.org/maps-for-europe/euroregionalmap/) dataset. Remember to include the **© EuroGeographics 2022** attribution.
  
- Code is licensed under [EUPL 1.2](https://github.com/eurostat/Nuts2json/blob/master/LICENSE).

### Contributions & Support

For any technical hurdles or collaborative efforts:

- Raise an [issue](https://github.com/eurostat/euronym/issues/new).
- Fork the project for enhancements.
  
### Disclaimer

While using or modifying this tool, remain cognizant of geopolitical nuances and boundaries as specified in the main documentation.

---

**Remember**: The given details are vital for any developer to comprehend, implement, or modify the Euronym tool. Ensure clarity and accuracy while handling the tool and its data.