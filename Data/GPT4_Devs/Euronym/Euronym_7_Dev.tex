Sure! Below are the adapted versions of the provided Euronym documentation for the defined audiences.

---

### Developer Version:

#### Euronym

Euronym is a dynamic label placement system specifically tailored for multi-scale web mapping applications in Europe. Through a sophisticated [label placement algorithm](https://en.wikipedia.org/wiki/Automatic_label_placement), it ensures that labels dynamically adjust in terms of density, selection, and size based on zoom levels and their significance.

![Euronym in Action](/docs/overview.gif)

#### Integration:
For developers looking to leverage Euronym's capabilities, it's embedded in the [Gridviz](https://github.com/eurostat/gridviz/blob/master/docs/reference.md#showing-labels) library. An implementation [example is available here](https://eurostat.github.io/gridviz/examples/labels_.html).

#### API Details:

- **Base URL:** `https://raw.githubusercontent.com/euronym/Nuts2json/master/pub/v2`
- **URL Pattern:** `/<ENCODING>/<RESOLUTION>/<GEO>.csv`
  
For specific use cases, a custom base URL can be set up as explained [here](#own-deployment).

##### Parameters:

| Parameter | Values | Description |
| ------------- | ------------- |-------------|
| `ENCODING` | `UTF`, `ASCII` | Preferred encoding. Default is `UTF`. |
| `RESOLUTION` | `20`, `50`, `100`, `200` | Determines the zoom level of the visualization in *m/pixel*. Smaller values result in denser labels and larger file size. |
| `GEO` | Multiple options available. `EUR` covers the entire dataset. | Geographic entity code. |

For extended parameters or more granular controls, please [open an issue here](https://github.com/eurostat/euronym/issues/new).

#### Data Format:
The data is structured in a CSV format:

| Column | Purpose |
| ------------- | ------------- |
| `name` | Text to be displayed on the map. |
| `lon` & `lat` | Geographical coordinates (in ETRS89 or WGS84 format). |
| `rs` & `r1` | Resolution boundaries for display adjustments. |
| `cc` | Country code when `GEO` is set to `EUR` for country-based filtering. |

Developers should note:
- Resolutions are *m/pixel* - representing pixel size in ground meters.
- Based on *rs* and *r1*:
  - Labels shouldn't display for resolutions above *rs*.
  - Between *rs* and *r1*, labels display at *1em* size.
  - Above *r1*, labels display at an exaggerated *1.5em* size.

#### Dependencies:
Euronym's functionality banks on these datasets:
- [EuroRegionalMap](https://eurogeographics.org/maps-for-europe/euroregionalmap/)
- [European commission town names table](https://ec.europa.eu/regional_policy/en/information/maps/urban-centres-towns)

The code transformation process is delineated [here](https://github.com/eurostat/euronym/tree/main/src/), which heavily utilizes the [GeoTools library](https://www.geotools.org).

#### Deployment:
For a custom Euronym deployment:
1. Clone the repository: `git clone [repository-link]`
2. Publish the `pub` directory on your preferred web server.
3. Replace the base URL with your domain, e.g., `https://www.mydomain.eu/path/to/pub/v2`.

This allows the customization of API elements as per your requirements.

#### Version:
2.0

#### Licensing:
The code is under the [EUPL 1.2 license](https://github.com/eurostat/Nuts2json/blob/master/LICENSE). Always ensure the inclusion of the **© EuroGeographics 2022** attribution when using the EuroRegionalMap dataset.

#### Support & Contribution:
For technical inquiries, debugging, or enhancement suggestions, please [open an issue](https://github.com/eurostat/euronym/issues/new). You're also encouraged to fork or star the project.

#### Copyright & Disclaimer:
Input datasets have their own copyrights. Users must ensure compliance, especially when using the [EuroRegionalMap dataset](https://eurogeographics.org/maps-for-europe/euroregionalmap/). The maps and representations are not endorsements by the European Union regarding the legal status of any entity.

--- 

For the non-developer audience, the documentation would be more focused on its usability and purpose rather than the technical intricacies. If you would like an adapted version for a non-developer audience, please let me know.