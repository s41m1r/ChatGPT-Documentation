### Developer Version:

---

## **Euronym: Multi-scale Web Mapping Label Placement for Europe**

Get adaptive place name labels for your web mapping application, leveraging a label placement algorithm that is specially designed for varying zoom levels and label importance.

![Euronym Overview](/docs/overview.gif)

---

### **Getting Started**

1. **Integrate with Libraries**: Euronym can be integrated with the [Gridviz library](https://github.com/eurostat/gridviz/blob/master/docs/reference.md#showing-labels). For an implementation reference, see the provided [example](https://eurostat.github.io/gridviz/examples/labels_.html).

2. **API Endpoint**: Use the API to fetch label data:
    - **Base URL**: `https://raw.githubusercontent.com/euronym/Nuts2json/master/pub/v2`
    - **URL Structure**: `/<ENCODING>/<RESOLUTION>/<GEO>.csv`
    
    *Note*: You can deploy and host your own version. Clone the repository, publish the `pub` folder on your server, and update the base URL accordingly. Refer to the [own deployment](#own-deployment) section for details.

---

### **Technical Specifications**

- **Parameters**:
    - `ENCODING`: Encoding format. Options: `UTF` (default), `ASCII`.
    - `RESOLUTION`: Level of detail, represented in *m/pixel*. Options: `20`, `50`, `100`, `200`. *Tip*: A lower value results in a denser label, increasing the file size.
    - `GEO`: Geographic entity code. For all data, use `EUR`. For specific geographic entities, refer [here](https://github.com/eurostat/euronym/tree/main/pub/v2/UTF/20).

- **Data Format**:
    - The response is a CSV with columns: `name`, `lon`, `lat`, `rs`, `r1`, `cc`.
    - Longitude and latitude values are in ETRS89 ([EPSG 4258](https://spatialreference.org/ref/epsg/etrs89/)). This can be considered equivalent to WGS84.

- **Rendering Guidelines**:
    - For resolutions > *rs*: Do not show the label.
    - For resolutions within [*rs*, *r1*]: Display the label at *1em* size.
    - For resolutions > *r1*: Display the label at *1.5em* size.

---

### **Dependencies & Sources**

Euronym is built upon:
- [EuroRegionalMap](https://eurogeographics.org/maps-for-europe/euroregionalmap/) from [EuroGeographics](https://eurogeographics.org/).
- [European commission town names table](https://ec.europa.eu/regional_policy/en/information/maps/urban-centres-towns).
    
The data transformation leverages the [GeoTools library](https://www.geotools.org). Explore the transformation process [here](https://github.com/eurostat/euronym/tree/main/src/).

---

### **Additional Resources**

- **Version**: 2
- **License**: Data: [EuroGeographics Open Data Licence](https://www.mapsforeurope.org/licence). Code: [EUPL 1.2](https://github.com/eurostat/Nuts2json/blob/master/LICENSE).
- **Contribution & Support**: If you encounter any challenges or want to contribute, raise an issue [here](https://github.com/eurostat/euronym/issues/new), fork the project, or give it a star.

---

### **Disclaimers**

Remember to adhere to the licensing and copyright terms when using the datasets. Add the **© EuroGeographics 2022** attribution. The map designations and presentations do not express any legal positions or boundaries from the European Union's perspective.

---

*Ensure you review the documentation, dependencies, and disclaimers before implementing or modifying the tool.*