**Developer Version**:

**Amazon S3 Buckets: Deep Dive**

---

**1. Introduction:**
Amazon S3 uses the concept of 'buckets' to manage data storage. Buckets are similar to directories or folders, though they exist at the root level of Amazon S3. Each AWS account can create up to 100 buckets, but this limit can be modified by interacting with the Service Quotas console.

**2. Addressing Objects:**
Objects, which can be anything from photos to documents, are stored inside these buckets. Objects have a unique address based on the bucket name and AWS Region. For instance, the object `photos/puppy.jpg` in the bucket `DOC-EXAMPLE-BUCKET` hosted in the US West (Oregon) Region has this URL:  
`https://DOC-EXAMPLE-BUCKET.s3.us-west-2.amazonaws.com/photos/puppy.jpg`.

**3. Interacting with AWS Resources:**
Both buckets and objects are treated as AWS resources. AWS offers S3 APIs to manage these resources. Whether you're aiming to create a new bucket or upload objects into one, the Amazon S3 API should be your go-to. While the AWS Console does offer GUI-based interactions, it's merely a frontend that sends API requests to Amazon S3.

**4. Global Bucket Uniqueness:**
Amazon S3 operates on a global namespace for bucket names. This means the bucket name you choose should be globally unique across all AWS accounts and all AWS Regions within a partition. Currently, AWS uses three primary partitions:

- `aws`: Standard Regions.
- `aws-cn`: Exclusive to China Regions.
- `aws-us-gov`: Pertains to AWS GovCloud (US).

It's crucial to note that once a bucket name is taken, it remains occupied in the specific partition unless the owning account deletes it.

**5. Naming Conventions:**
While there are conventions around naming your buckets (provided in the "Bucket naming rules"), developers shouldn't rely on these conventions for ensuring availability or security verification.

**6. Bucket Region Selection:**
Buckets are created in a specific AWS Region of your choosing. This region selection can affect latency, costs, and even compliance with certain regulations. As a best practice, select a region closest to the majority of your user base or in compliance with data regulations specific to your use-case. For complete details on available regions, refer to the AWS General Reference under "Regions and Endpoints".

**7. Implementation Details:**
When implementing S3 bucket operations in your code:

- Ensure global uniqueness in bucket names.
- Check the availability of a bucket name before attempting to create one.
- Keep latency in mind. Choose a region that optimizes for your primary user base.
- Familiarize yourself with AWS SDKs available in multiple languages to interact with the S3 APIs.
- Regularly check for updates in S3 APIs for new features or any deprecations.

For further details on working with objects inside the buckets, developers should consult the "Amazon S3 objects overview".